\section{Introducción}
La Inteligencia Artificial (IA) ha tenido un crecimiento exponencial en los últimos años debido a la capacidad de cómputo asequible y de alto rendimiento, así como a la disponibilidad de grandes volúmenes de datos. Se destaca el Procesamiento de Lenguaje Natural (PLN) como una de las áreas de estudio de la IA que utiliza modelos estadísticos, aprendizaje automático y aprendizaje profundo para permitir que las computadoras entiendan e interpreten el lenguaje humano.\\
\indent Una de las tantas aplicaciones prácticas del PLN son los chatbots, programas que imitan la conversación humana y son capaces de interactuar con personas y responder adecuadamente a sus preguntas. Los chatbots son accesibles, eficientes y de alta disponibilidad, lo que permite que diferentes industrias se beneficien de ellos, como el comercio electrónico, los seguros y el cuidado de la salud.\\
\indent El trabajo presenta un chatbot específicamente diseñado para el sector de la educación, con el objetivo de permitir que los estudiantes realicen consultas sobre la Facultad de Ingeniería y obtengan respuestas mediante técnicas de coincidencia de patrones.

\subsection{Objetivo General}

Desarrollar e implementar un Chatbot utilizando algoritmos de Inteligencia Artificial (IA).

\subsection{Objetivos Específicos}
\begin{itemize}
\item Comparar distintas Tecnologías para abordar el problema.
\item Orientar el chatbot a dudas comunes de los estudiantes de la Facultad de Ingeniería en atención al alumno.
\item Recopilar, procesar y filtrar preguntas frecuentes para contar con un dataset
propio.
\item Seleccionar, entrenar y probar el algoritmo utilizando el dataset generado.
\item Implementar el chatbot para su uso por estudiantes de la Facultad de Ingeniería
\item Agregar una interfaz, propia o a una ya existente para el uso por los alumnos.

\end{itemize}

\subsection{Alcance y limitaciones}

Se pretende desarrollar un Chatbot que utilice algoritmos de inteligencia artificial para responder a preguntas frecuentes de los alumnos de la Facultad de Ingeniería de la
UNA. El entrenamiento de la IA se llevará a cabo utilizando un dataset propio en
conjunto con otros ya existentes. Una previa comparativa entre distintas librerías
y plataformas es necesaria para escoger la mejor solución al objetivo planteado.