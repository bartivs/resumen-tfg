%TODO: agregar contenido en el mismo formato
\twocolumn[
	\begin{@twocolumnfalse}
		\maketitle
		%%\hline \\
		%\hspace{-0.5cm} \rule{\linewidth}{0.1mm}\\
		%\begin{table}
		\centering
		\begin{tabular}{p{6cm} p{10.3cm}} %4.3 12
			\\ \hline

			\vspace{-0.1cm}
			\uline{Palabras clave:\hfill}
			\begin{itemize}
				\setlength\itemsep{-0.2em}
				\item Chatbot
				\item Procesamiento del Lenguaje natural
				\item Inteligencia Artificial
				\item Dudas Académicas
			\end{itemize}
			 & \vspace{-0.1cm}

			\uline{Resumen\hfill}
			Este trabajo de Fin de Grado se presenta una comparativa de varias
			tecnologías disponibles para el desarrollo de chatbots, teniendo en cuenta
			sus ventajas y
			desventajas. Después de este análisis, se seleccionó el entorno de
			desarrollo proporcionado por
			Rasa Open Source para realizar un examen más exhaustivo de esta tecnología.

			Como resultado, se implementó un chatbot que fue entrenado con datos propios y recolectados
			mediante pruebas realizadas con grupos de estudiantes de la Facultad de
			Ingeniería cuyos
			comentarios fueron valiosos mejorar la calidad del chatbot. Así como
			también se publicó la integro
			la solución a la aplicación de Telegram.
			\thispagestyle{firststyle}

			\text{}
			\\
			\uline{Keywords\hfill}
			\begin{itemize}
				\setlength\itemsep{-0.2em}
				\item Chatbot
				\item Natural Language Processing
				\item Artificial Intelligence
				\item Academic Queries
			\end{itemize}
			 &
			\uline{Abstract \hfill}

			This thesis presents a comparison of several technologies available for the
			development of chatbots, taking into account their advantages and disadvantages.
			technologies available for the development of chatbots, taking into account
			their advantages and disadvantages.
			disadvantages. After this analysis, the development environment provided by
			Rasa Open Source was selected for a more thorough examination of this technology.
			Rasa Open Source was selected for a more thorough examination of this
			technology.

			As a result, a chatbot was implemented and trained with its own data and data collected
			through tests conducted with groups of students from
			through testing with groups of students from the School of Engineering
			whose feedback was valuable to improve the quality of the chatbot.
			feedback was valuable to improve the quality of the chatbot. As well as
			publishing the integro
			the solution to the Telegram application
			\\ \hline

		\end{tabular}

		%\hspace{-0.5cm} \rule{\linewidth}{0.1mm}\\
	\end{@twocolumnfalse}
]
