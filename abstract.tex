%TODO: agregar contenido en el mismo formato
\twocolumn[
	\begin{@twocolumnfalse}
		\maketitle
		%%\hline \\
		%\hspace{-0.5cm} \rule{\linewidth}{0.1mm}\\
		%\begin{table}
		\centering
		\begin{tabular}{p{6cm} p{10.3cm}} %4.3 12
			\\ \hline

			\vspace{-0.1cm}
			\uline{Palabras clave:\hfill}
			\begin{itemize}
				\setlength\itemsep{-0.2em}
				\item Sistemas de Determinación y Control de Actitud (ADCS)
				\item Magnetorquer
				\item Rueda de reacción
				\item Jaula de Helmholtz
			\end{itemize}
			                                                        & \vspace{-0.1cm}
			\uline{Resumen\hfill}
			\thispagestyle{firststyle}

			El Sistema de Determinación y Control de Actitud (ADCS) es un subsistema
			crucial de un satélite CubeSat, ya que proporciona precisión y estabilidad de puntería para las
			cargas útiles y las antenas, utilizando sensores para determinar y actuadores para controlar la
			actitud. El presente Trabajo Final de Grado presenta el diseño e implementación de un modelo
			conceptual ADCS para un CubeSat de 1-U. Son implementados,  el algoritmo del filtro Madgwick para
			determinar la actitud a partir de una representación de cuaterniones, y el método de compensación
			PID para controlar a los actuadores. El primer paso para diseñar es definir los marcos de
			referencia, la representación de la actitud y los algortimos ADCS. Así también se diseñan los
			distintos dispositivos que componen el hardware del ADCS con la estructura del CubeSat, que
			finalmente son implementados mediante la integración completa, realizando validaciones de
			funcionamiento a través de simulaciones y pruebas experimentales en un banco de pruebas construido,
			consistiendo en una jaula de Helmholz de 1 eje y la denominada prueba piñata. El diseño e
			implementación del modelo conceptual del ADCS con la jaula de Hemholtz apoyan a el desarrollo de
			capacidades y la educación en ingeniería espacial en las universidades paraguayas

			\text{}
			\\
			\uline{Keywords\hfill}
			\begin{itemize}
				\setlength\itemsep{-0.2em}
				\item Attitude Determination and Control Systems (ADCS)
				\item Magnetorquer
				\item Reaction wheel
				\item Helmholtz cage
			\end{itemize} &
			\uline{Abstract \hfill}

			The Attitude Determination and Control System (ADCS) is a crucial subsystem
			of a CubeSat satellite, providing precision and pointing stability for payloads and antennas, using
			sensors to determine and actuators to control attitude. This Final Degree Project presents the
			design and implementation of an ADCS conceptual model for a 1-U CubeSat. They are implemented, the
			Madgwick filter algorithm to determine the attitude from a representation of quaternions, and the
			PID compensation method to control the actuators. The first step in designing is to define the
			frames of reference, the attitude representation, and the ADCS algorithms. Also the different
			devices that compose the ADCS hardware are designed with the CubeSat structure, which are finally
			implemented through complete integration, performing operational validations through simulations
			and experimental tests on a built test bench, consisting of a cage. Helmholtz 1 axis and the
			so-called piñata test. The design and implementation of the ADCS conceptual model with the
			Helmholtz cage support capacity development and education in space engineering in Paraguayan
			universities.

			\\ \hline

		\end{tabular}

		%\hspace{-0.5cm} \rule{\linewidth}{0.1mm}\\
	\end{@twocolumnfalse}
]
